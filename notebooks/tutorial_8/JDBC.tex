\documentclass[11pt]{article}
%\usepackage[margin=0.5in]{geometry}
%\usepackage[hmargin=0.25in, vmargin=0.75in]{geometry}
\usepackage[hmargin=0.5in, vmargin=0.5in]{geometry}

\newcommand{\urule}[2][.5pt]{\rule[-2pt]{#2}{#1}} % "underline" rule
\newif\ifsolutions % starts out false by default
%\solutionstrue % comment out to format solutions
\newif\ifresults % starts out false by default
%\resultstrue % comment out to format solutions

\pagestyle{empty}

\begin{document}

%~~~ \vspace{-2cm}

\noindent
University of Toronto\\
{\sc CSC}343,
Fall 2018\\[10pt]
{\huge \bf JDBC Exercise}\\[10pt]

% --------- Important not to self!!! ----------
% In the jelly-beans.sql code that students get, don't set the search_path.
% Otherwise, the JDBC code will not find the table, because it doesn't set the search path.


\noindent
Follow these steps to connect a Java program to your postgreSQL database and run some
queries from Java.
\textbf{See the other side for the most common errors students make.}

\begin{enumerate}
\item  %----------
Log on to \verb+dbsrv1.cdf.utoronto.ca+
\item  %----------
Create a directory where you'd like to do this work and \verb+cd+ to it.
\item  %----------
Get a copy of a file, which defines a small dataset (\verb+jelly-beans.sql+)
and some java code (\verb+Example.java+) that you will use today:

\begin{verbatim}
    cp ~csc343h/fall/public_html/in_class/w6/code/* .
\end{verbatim}
(You really do need that period at the end.)
\item  %----------
Start postgreSQL: 
\begin{verbatim}
    psql csc343h-<your username>
\end{verbatim}
\item  %----------
Import jelly-beans.sql into your database:
\begin{verbatim}
    \i jelly-beans.sql
\end{verbatim}
\item  %----------
Use \verb+\d+ to see what is defined and a \verb+SELECT+ query to examine the contents of 
table \verb+guesses+.
Then use \verb+\q+ to exit psql.
\item  %----------
Get your bearings in the code.  Find the parts where it:
\begin{enumerate}
\item
Establishes a connection to the database.
\item
Runs a query to find all the guesses by kids under 10, and iterates through the results to print them.
\item
Builds a query to find all the guesses by a particular person, but with a {\it placeholder} for
the person's name.  The program then substitutes in a name entered by the user and goes on to
run the query and print the results.
\end{enumerate}
\item  %----------
Edit \verb+Example.java+ to replace \verb+dianeh+ with your cdf userid on these two lines: 
\begin{verbatim}
    url = "jdbc:postgresql://localhost:5432/csc343h-dianeh";
    conn = DriverManager.getConnection(url, "dianeh", "");
\end{verbatim}
\item  %----------
Compile the Java code:
\begin{verbatim}
    javac Example.java
\end{verbatim}
\item  %----------
Run the compiled code:
\begin{verbatim}
    java -cp /local/packages/jdbc-postgresql/postgresql-9.4.1212.jar: Example
\end{verbatim}
The path to the \verb+jar+ file is hard to type.
Tip: Use filename completion!
(And you really do need that colon.)
\end{enumerate}



\section*{Additional Exercises}

\begin{enumerate}
\setcounter{enumii}{10}
\item  
Prompt the user for an age,
read the age,
and then print the average guess
among those from people who are at least that age.
\item
Create an array of exactly the right size to hold every name in the table of guesses
(without storing duplicates),
and then put those name into the array.
\end{enumerate}
% \item  %----------
% The provided example runs a query on the database, but it doesn't change anything. In this step, go back to accessing the database
% again through postgreSQL as you did in step 4. You can skip step 5 this time because you don't need to import the table again. Use 
% a \verb+SELECT+ statement to see that the database has not changed. Now use an \verb+INSERT+ statement to add a new guesser to the table as number 12.
%
%
%\item %-----------
%Use a \verb+DELETE+ statement to remove the new guesser that you just added.
%
%\item %-------------
%{\bf Helpful Hint:} Before you start this next step, open a second window on cdf. Leave the postgreSQL prompt open in the first window, since 
%we will be coming back to examine the database after doing the next step.
%  
%Add to \verb+Example.java+ so that it inserts guesser number 12. You don't need to ask the
%user for the name or the number of guesses, just hardcode these values into your code. Compile and run your updated Java code. See
%below for common errors. The last error, where you already have a tuple with number 12,  might surprise some of you. If you
%correctly used \verb+DELETE+ to remove this tuple, why is it still there? This will happen if you run your program twice and may even happen if your
%program failed on an earlier run. Depending on \emph{when} the failure happened, the tuple may have already been inserted. Repeat step 
%12 to delete the tuple again and then rerun your program.
%
%\item %--------------
%Once you have successfully run your Example code, use a \verb+SELECT+ statement from the postgreSQL prompt to confirm that the new 
%guesser was added. If you didn't already experience the error discussed above, rerun your program and see it now. Play around with
%editing and viewing the database from the postgreSQL prompt in between runs of your program. 

\vfill
\section*{Common errors}
\begin{itemize}
\item If you forget the colon:
\begin{verbatim}
wolf% java -cp /local/packages/jdbc-postgresql/postgresql-9.4.1212.jar Example
Error: Could not find or load main class Example
\end{verbatim}
\item If you try to run the code on any machine other than dbsrv1:
\begin{verbatim}
wolf% java -cp /local/packages/jdbc-postgresql/postgresql-9.4.1212.jar: Example
SQL Exception.<Message>: Connection refused. Check that the hostname and port are correct 
and that the postmaster is accepting TCP/IP connections.
\end{verbatim}
\item If you try to use an \verb|executeQuery| statement to do an insert. You want to use \verb|executeUpdate| instead. 
\begin{verbatim}
SQL Exception.<Message>: No results were returned by the query.
\end{verbatim}
% This is an error that you could see inside psql so doesn't belong with this list of JDBC-specific errors.
%\item If you try to use your Example.java program to insert a tuple for guesser number 12 when there is already one in the table:
%\begin{verbatim}
%SQL Exception.<Message>: ERROR: duplicate key value violates unique constraint 
%          "guesses_pkey"
%  Detail: Key (number)=(12) already exists.
%\end{verbatim}
\end{itemize}

\end{document}






